\documentclass[12pt]{article}
%\usepackage{fullpage}
\usepackage{epic}
\usepackage{eepic}
\usepackage{paralist}
\usepackage{graphicx}
\usepackage{algorithm,algorithmic}
\usepackage{tikz}
\usepackage{xcolor,colortbl}
\usepackage{wrapfig}
\usepackage{amsthm}
\usepackage{amsmath}
\usepackage{listings}


%%%%%%%%%%%%%%%%%%%%%%%%%%%%%%%%%%%%%%%%%%%%%%%%%%%%%%%%%%%%%%%%
% This is FULLPAGE.STY by H.Partl, Version 2 as of 15 Dec 1988.
% Document Style Option to fill the paper just like Plain TeX.

\typeout{Style Option FULLPAGE Version 2 as of 15 Dec 1988}

\topmargin 0pt
\advance \topmargin by -\headheight
\advance \topmargin by -\headsep

\textheight 8.9in

\oddsidemargin 0pt
\evensidemargin \oddsidemargin
\marginparwidth 0.5in

\textwidth 6.5in
%%%%%%%%%%%%%%%%%%%%%%%%%%%%%%%%%%%%%%%%%%%%%%%%%%%%%%%%%%%%%%%%

\pagestyle{empty}
\setlength{\oddsidemargin}{0in}
\setlength{\topmargin}{-0.8in}
\setlength{\textwidth}{6.8in}
\setlength{\textheight}{9.5in}


\def\ind{\hspace*{0.3in}}
\def\gap{0.1in}
\def\bigap{0.25in}
\newcommand{\Xomit}[1]{}
\renewcommand\thesection{(\alph{section})}


\begin{document}

\setlength{\parindent}{0in}
\addtolength{\parskip}{0.1cm}
\setlength{\fboxrule}{.5mm}\setlength{\fboxsep}{1.2mm}
\newlength{\boxlength}\setlength{\boxlength}{\textwidth}
\addtolength{\boxlength}{-4mm}
\begin{center}\framebox{\parbox{\boxlength}{{\bf
CS 4820, Fall 2018 \hfill Homework 1, Problem 3}%\\
%Collaborators:
}}
\end{center}
\vspace{5mm}
In this problem, we need to choose a port among each ship's schedule: the port that the corresponding ship will stay in the rest of the month. Similar to the propose-respond problem discussed in the class, we can simply set up a similar scenario, where ships are regarded as ``proposer" while ports are regarded as ``respondent". Each ship ranks each port in the chronological order as presented in the schedule; each port ranks each ship in the reverse chronological order as presented in the schedule. According to the class, there will always be a stable matching. Hence, now it is simple to prove such a set of truncations can always be found given the stable matching exists.\\\\
\textbf{Proof} \\
We can use contradictory: A ship $s_i$ gets to port $p_i$ after ship $s_j$ has already staying in $p_i$. In this case, since $s_i$ prefers $p_i$ to the port in its schedule (because it passes $p_i$ to its destination), and $p_i$ prefers $s_i$ to $s_j$ (rever chronological order), which indicates that $s_i$ should be matched with $p_i$ (i.e., $s_j$ should not be staying there), so it contradicts the stable matching assumption. Hence, we conclude that such a set of truncations can always be found.\\\\  
The algorithm is stated as follow:
\begin{algorithm}[H]
\caption{Algorithm for problem 3}
Regard ships $S$ as ``proposers" \\
Regard ports $P$ as ``respondents"\\
Each ship ranks each port in the chronological order as presented in the schedule\\
Each port ranks each ship in the reverse chronological order as presented in the schedule\\
Initially all $s\in S$ and $p\in P$ are unmatched\\
\textbf{While} there is a ship $s$ that is unmatched and has not ``proposed" to every port\\
\hspace*{10mm}Choose such a ship $s$\\
\hspace*{10mm}Let $p$ be the highest-ranked port in $s$'s preference list which $s$ has not yet ``proposed"\\
\hspace*{10mm}\textbf{If} $p$ is not matched then\\
\hspace*{20mm}$(s,p)$ are matched\\
\hspace*{10mm}\textbf{Else} $p$ is currently matched to $s'$\\
\hspace*{20mm}\textbf{If} $p$ prefers $s'$ to $s$ then\\
\hspace*{30mm}$s$ remains unmatched\\
\hspace*{20mm}\textbf{Else} $p$ prefers $s$ to $s'$\\
\hspace*{30mm}$(s,p)$ become matched\\
\hspace*{30mm}$s'$ becomes unmatched\\
\hspace*{20mm}\textbf{End if}\\
\hspace*{10mm}\textbf{End if}\\
\textbf{End while}\\
Return the set S of matched pairs.
\end{algorithm}
\textbf{Time complexity analysis:}\\
The complexity of one round is $O(1)$\\
Observing each ship ``proposes" to each port no more than once\\
$n$ ships $\times$ $n$ ports = $O(n^2)$

\end{document}
